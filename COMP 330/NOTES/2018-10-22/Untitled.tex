\documentclass[11pt]{article}

\usepackage[top = 1in, bottom = 1in, left = 0.8in, right = 0.8in, headheight = 14pt]{geometry}
\usepackage{kpfonts}
\usepackage{dsfont}
\usepackage[T1]{fontenc}
\usepackage{fancyhdr}
\usepackage{wrapfig}
\usepackage{tikz}
\usetikzlibrary{positioning}
\usepackage[framemethod = tikz]{mdframed}

\let\openbox\relax
\usepackage{amsmath}
\usepackage{amssymb}
\usepackage{amsthm}
\usepackage{mathtools}
\usepackage{enumitem}
\usepackage{array}
\usepackage{algorithm}
\usepackage[noend]{algpseudocode}

\pagestyle{fancy}
\fancyhf{}
\fancyhead[L]{\textbf{COMP 330 notes}}
\fancyhead[R]{Yuhao Wu 260711365}
\fancyfoot[C]{\thepage}
\renewcommand{\headrulewidth}{2pt}

\newcommand{\paren}[1]{\left( #1 \right)}
\newcommand{\sqrbrak}[1]{\left[ #1 \right]}
\newcommand{\curbrak}[1]{\left\{ #1 \right\}}
\newcommand{\lrangle}[1]{\langle #1 \rangle}
\newcommand{\LopRcl}[1]{\left(\left. #1 \right]\right.}
\newcommand{\LclRop}[1]{\left.\left[ #1 \right.\right)}
\newcommand{\eqabove}[1]{\overset{#1}{=}}
\newcommand{\ith}[1]{#1^{\text{th}}}

\DeclarePairedDelimiter{\abs}{\lvert}{\rvert}
\DeclareMathOperator{\N}{\mathbb{N}}
\DeclareMathOperator{\Z}{\mathbb{Z}}
\DeclareMathOperator{\E}{\mathbb{E}}
\DeclareMathOperator{\R}{\mathbb{R}}
\DeclareMathOperator{\sigmafield}{\mathcal{S}}
\DeclareMathOperator{\Borel}{\mathcal{B}\paren{\R}}
\DeclareMathOperator{\probsymb}{\mathbb{P}}
\newcommand{\vbar}{\,\middle|\,}
\newcommand{\func}[2]{#1\paren{#2}}
\newcommand{\prob}[1]{\mathbb{P}\paren{#1}}
\newcommand{\diffop}[1]{\,\mathrm{d}#1}
\newcommand{\indicator}[2]{\func{\mathds{1}_{#1}}{#2}}
\newcommand{\DF}{\textsc{df}\ }
\newcommand{\DFnospace}{\textsc{df}}
\newcommand{\PMF}{\textsc{pmf}\ }
\newcommand{\PMFnospace}{\textsc{pmf}}
\newcommand{\PDF}{\textsc{pdf}\ }
\newcommand{\PDFnospace}{\textsc{pdf}}
\newcommand{\GammaFunc}[1]{\func{\Gamma}{#1}}

\newenvironment{solution}{\renewcommand\qedsymbol{$\blacksquare$}\begin{proof}[Solution]}{\end{proof}}


\setlength{\parindent}{0pt}
\begin{document}
\title{COMP 330}
\date{2018-10-22}
\author{Name: Yuhao Wu\\
ID Number: 260711365
}
\maketitle

	
\textbf{Oct 22 \quad Lecture 17}\\\\
Three important facts
\begin{itemize}
	\item Acceptance only happens at the end of the input
	\item A PDA can't decide to jam when there are moves possible
	\item When there are choices, PDA can select any choices it wants
\end{itemize}

\textbf{REMARK:}
\begin{itemize}
	\item PDAs are equivalent to CFLs
	\item DPDAs are equivalent to DCFLs
	$$ \delta: Q \times \Sigma_{\varepsilon} \times \Gamma_{\varepsilon} \to (Q \times \Gamma_{\varepsilon})\ \ or \ \ \emptyset $$
\end{itemize}

\textbf{REMARK:}\\
For every $q \in Q, a \in \Sigma,  x\in \Gamma$, \textbf{exactly one } of $\delta(q, a, x),\ \delta(q, a, \varepsilon ),\ \delta( q, \varepsilon, x ),\  \delta (q, \varepsilon, \varepsilon ) $ is non-empty\\
\\
\textbf{Counter-Example such that the intersection of 2CFL is not CFL}
$$\{ a^n\ b^n\ c^m | n, m \geq 0  \} \cap \{ a^m\ b^n\ c^n | n, m \geq 0  \} = 
\{ a^n b^n c^n \} $$
\\
\textbf{Algorithms for CFLs:}
$$Is\ L(G) = \emptyset\ ? \quad S \to aS$$
We say that a NT x is generating if $x \xrightarrow{*} w \in \Sigma^* (\ or\ T^*)$\\
\\
\textbf{Theorem : } $L(G) \neq \emptyset \iff S$ is generating\\

\newpage
\begin{algorithm}
\caption{Check if it is generating}
\begin{algorithmic}[1]
\Procedure{algorithm}{$\quad$}
\State initialize $GEN = \emptyset$
\State Put all terminal symbols in $GEN$
\While{ there is still some changes happening}\Comment{repeat until no change happens}
\State for each rule $x \to \alpha$, check if every symbol of $\alpha$ belongs to $GEN^*$
\State if so, add $x$ to $GEN$
\EndWhile\label{euclidendwhile}
\If{$S \in Gen$}
	\State return $L(G) \neq \emptyset$ 
\EndIf
\EndProcedure
\end{algorithmic}
\end{algorithm}

\textbf{EXAMPLES[1]:}\\
$S \to aS \quad GEN = \{ a \}$\\
Now we have just one rule $S \to \alpha \ where\  \alpha = aS$, as not all of $a, S$ belong to $GEN^*$, we can now end the loop, and $S \notin GEN$, we can conclude $L(G) = \emptyset$\\
\\
\textbf{EXAMPLES[2]:}\\
$S \to bA | aB \quad \quad A \to bAA|aS|a \quad\quad B\to aBB|bS|b$\\
$[1]:\quad GEN = \{a, b\} \quad\quad [2]:\quad GEN = \{ a, b, A, B, S \} \implies L(G) \neq \emptyset $
\\
\\
There are more complex algorithms to check if $L(G) = \infty$
\\
\\
\textbf{REMARK:}
\begin{itemize}
	\item The complement of CFL may not be a CFL
	\item The complement of DCFL is a DCFL
\end{itemize}





\textbf{OCT.24\quad\quad Lecture 18}\\
Question: $w \in L(G)$ ? \\
We can convert $G$ to Chomsky Normal Form $X(non-T) \to AB \quad \quad X \to a (terminal)$\\
Using Dynamic Programming $O(n^3)$\\
\\
(Cocke-Younger-Kasami Algorithm)
\\
\\
\\
\textbf{PDA: 2 distinct notions of accpetance}
\begin{itemize}
	\item Accept by accept states
	\item Accept by empty stack: at the end of the string is the stack empty?\\
	There are no accept states.
\end{itemize}
Both of them are equally powerful.


\newpage
\textbf{OCT.26\quad\quad Lecture 19}\\
\\
\\
\textbf{PUMPING LEMMA:}\\
Let $L$ be a $CFL$.\\
 $\exists p \geq 0$ such that $\forall s \in L$ with $|s|\geq p$, $\exists u,v,w,x,y \in \Sigma^*$ where $s=uvwxy$ such that:
 \begin{itemize}
 	\item $|vx|>0$  (i.e., $v$ and $x$ can't both be the empty string);
 	\item $|vwx| \leq p$
	\item $\forall i\geq 0,\ uv^iwx^iy \in L$
 \end{itemize}
 As with regular languages, we can use the contrapositive of the pumping lemma to prove that a language is not context-free. Here's a formal statement of the contrapositive:\\\\
 Fix some $L$\\
 $\forall p \geq 0$ such that $\exists s \in L$ with $|s|\geq p$, $\forall u,v,w,x,y \in \Sigma^*$ where $s=uvwxy$ such that:
  \begin{itemize}
 	\item $|vx|>0$  (i.e., $v$ and $x$ can't both be the empty string);
 	\item $|vwx| \leq p$
	\item $\exists i\geq 0,\ uv^iwx^iy \notin L$
 \end{itemize}
Then we can say that $L$ is not a CFL\\
\\
\textbf{EXAMPLE [1]} $L = \{ a^n\ b^n\ a^n\ |\ n\geq 0 \}$
\begin{itemize}
	\item Demon: choose $P$
	\item I choose $s = a^p\ b^p\ a^p \in L$
	\item Demon chooses any $ u,v,w,x,y \in \Sigma^*$ where $s=uvwxy$
\end{itemize}
$$ \begin{matrix} \underbrace{ a, a, \ldots, a } \\ p \end{matrix} \quad \begin{matrix} \underbrace{ b, b, \ldots, b } \\ p \end{matrix} \quad
\begin{matrix} \underbrace{ a, a, \ldots, a } \\ p \end{matrix} $$

\begin{itemize}
	\item $v$  or $x$ straddles a block boundary. Choose $i=2$. Then, the a's and b's would be out of order in the resulting string.
	\item If $v$ and $x$ both contain only a's or b's, then $v$ and $x$ must be in the same block since $|vwx|<p$, choose $i = 2$
	\item $v$ has a's and $x$ has b's or $v$ has b's and $x$ has a's\\
		Then we can choose $i = 2$
	
\end{itemize}

\newpage

\textbf{EXAMPLE [2]} $L = \{\ ww | w \in \Sigma^*\ \}$ suppose $\Sigma$ has 2 letters.\\
Then, we define a new language 
$$\widehat{L} = L \cap a^*b^*a^*b^* = \{ a^m b^n a^m b^n | m, n \geq 0 \}$$
Suppose that $L$ is CFL then $\widehat{L} = CFL \cap \text{ regular language} = CFL$
\begin{itemize}
	\item Demon: choose $P$
	\item I choose $s = a^p\ b^p\ a^p\ b^p  \in L$
	\item Demon chooses any $ u,v,w,x,y \in \Sigma^*$ where $s=uvwxy$
\end{itemize}
$$ \begin{matrix} \underbrace{ a, a, \ldots, a } \\ p \end{matrix} \quad \begin{matrix} \underbrace{ b, b, \ldots, b } \\ p \end{matrix} \quad
\begin{matrix} \underbrace{ a, a, \ldots, a } \\ p \end{matrix} \quad
\begin{matrix} \underbrace{ b, b, \ldots, b } \\ p \end{matrix} $$

\textbf{NOTE:}
$\overline{L}$ is a CFL
$$S \to AB | BA | A | B \quad \quad A \to CAC | a \quad \quad B \to CBC|b \quad \quad C \to a | b$$
$$ \begin{matrix} \underbrace{ \ldots \ldots\ldots  } \\ n \end{matrix} \quad 
\begin{matrix} \underbrace{ a} \\ n + 1 \ position \end{matrix}
 \quad \begin{matrix} \underbrace{ \ldots \ldots\ldots } \\ n \end{matrix} \quad
\begin{matrix} \underbrace{ \ldots \ldots\ldots  } \\ m \end{matrix} \quad 
\begin{matrix} \underbrace{ b } \\ n + 1 \text{ position in the second half}\end{matrix}
 \quad \begin{matrix} \underbrace{ \ldots \ldots\ldots } \\ m \end{matrix} \quad $$
\newline
$$ L' = \{\ w\ w^{rec}\  | w \in \Sigma^* \}\quad\quad \textbf{this is CFL}$$
\\
\\
\\
\textbf{OCT 29 Lecture 20:}\\
\textbf{REMARK:}\\
\begin{itemize}
	\item CFL \textbf{are NOT} closed under complementation. \\
	Suppose $L_1$ and $L_2$ are context free. $L_1 \cap L_2$ is not necessarily context-free
	\item $L_1 \cup L_2$ we just need to add one rule $S \to S_1 | S_2$
	\item $L_1 \cdot L_2$ we just need to add one rule $S \to S_1  S_2$
\end{itemize}

\textbf{EXAPMLE [1]:} Show $L = \{ 0^i 1^j\ |\ j = i^2 \}$ is not a CFL
\begin{itemize}
	\item Demon: choose $P$
	\item I choose $s = 0^p\ 1^{p^2}  \in L$
	\item Demon chooses any $ u,v,w,x,y \in \Sigma^*$ where $s=uvwxy$
\end{itemize}
$$ \begin{matrix} \underbrace{ 0, 0, \ldots, 0 } \\ p \end{matrix} \quad \begin{matrix} \underbrace{ 1, 1, \ldots, 1 } \\ p^2 \end{matrix}$$
\begin{itemize}
	\item $vwx$: all zeros or all ones \quad pick $i \neq 1$
	\item $v\ or \ x$ overlaps the boundary. I pick $i = 2$
	\item $v$ : all 0's \quad $x$: all 1's 
	Either $|v| = 0 \ or \ |x| = 0 \ pick\ i = 2 \quad\quad if\ |v|, |u|\neq 0 \ \ \ pick\ i = 2$
\end{itemize}
\newpage
\textbf{Exercise [1]:} $L = \{ a^p \ |\ p \text{ is a prime} \}$
\\
\\
\textbf{Exercise [2]:} $L = \{ a^i\ b^j\ c^k \ |\ 0 < i < j < k \}$











\end{document}